\section{Introduction}
\label{Introduction}

This template is provided for the Research Project and Paper for the CIS*4150 Software Reliability and Testing course. It is based on the Springer Nature template, commonly used in academic journals and conferences. The template outlines how to structure your paper, include figures and tables, and cite references from the bibliography.

Once you have reviewed this document, replace the placeholder content with your own in the appropriate sections.

I hope you find this helpful!

Ed

The Introduction sets the stage for the research by providing a high-level overview of background information, defining the problem or research question, and explaining the study's objectives. It typically includes a couple key relevant papers to highlight gaps in current knowledge, justifies the need for the study, and outlines the paper's structure. The introduction should engage the reader by clearly explaining why the research is important and what the study aims to contribute to the field. It often ends with a brief summary of the research approach or hypothesis.


Generative AI, encompassing technologies such as Large Language Models (LLMs) like ChatGPT, and advanced image generation tools, is profoundly reshaping various facets of the digital and real world. As educators and researchers, it is essential to stay abreast of these advancements to enhance our teaching methods and pedagogical tools, thereby enriching both our own and our students' skillsets.

This research delves into the specific application of LLMs, focusing on ChatGPT, to assess its impact on developing academic critique skills among Computer Science undergraduates enrolled in a fourth-year Ubiquitous Computing course. The core objective of this study is to evaluate and discern the differences between student-authored critiques and those augmented by ChatGPT's assistance.

Through this investigation, we aim to highlight the potential of LLMs to bolster students' critical thinking and writing capabilities. This paper seeks to provide valuable insights into how AI tools can be seamlessly integrated within educational frameworks.  Additionally, we explore the transformative role of AI-driven tools in supporting student learning and enhancing the personalized and engaging nature of academic critique processes.

The remainder of this paper is structured as follows: The next section offers a background on Generative AI, focusing on the Transformer architecture, ChatGPT, its applications in education, and readability metrics. In Section~\ref{Methodology}, we outline the methods employed to investigate the effectiveness of ChatGPT as a collaborative tool for supporting technical critique writing. The findings from our quantitative and qualitative analysis are presented in Section~\ref{Findings}. Section~\ref{Discussion} reflects on these findings within a broader context, and Section~\ref{Conclusion} concludes the paper while highlighting potential future directions for this line of research.